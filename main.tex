\documentclass[15,a4paper]{report}
\usepackage[utf8]{inputenc}
\usepackage{blindtext}
\usepackage{titlesec}
\usepackage{amsfonts}
\usepackage{amsmath}
\usepackage{amssymb}
\usepackage[english]{babel}
\usepackage{amsthm}
\theoremstyle{definition}
\newtheorem{definition}{Definition}[section]
\newtheorem{theorem}{Theorem}[section]
\theoremstyle{remark}
\newtheorem*{remark}{Remark}
\newtheorem{corollary}{Corollary}[theorem]
\newtheorem{lemma}[theorem]{Lemma}
\usepackage[fontsize=12pt]{fontsize}

\pagenumbering{arabic}
\title{\textbf{\Huge{Number theory}}}
\author{\Large{Tanmay Joshi} \\
\Large{Mentor: Shourya Pandey}}
\date{Summer of Science 2021}

\begin{document}

\maketitle



\tableofcontents


\chapter*{Report overview}

\paragraph{}In this report I will be summarizing my readings in the first half of my summer of science- 2021 project on Number theory. The book I have referred to is  \textit{Elementary Number Theory - David M Burton}. We'll be going through some preliminaries and building up the required knowledge to understand some centuries old conjectures and also prove some historic theorems. Let us begin!


\chapter{Preliminaries}



\section{Constructing the integers}
\begin{definition}
    For a given set S and a given equivalence relation $\mathtt{\sim}$ on S, the equivalence class of a, denoted by [a], is given by: 
    \begin{math}
        [a] = \{ x \epsilon S | x \mathtt{\sim} a \}
    \end{math}
\end{definition}

\paragraph{} Let the equivalence relation $\mathtt{\sim}$ be defined on $\mathbb{N} X \mathbb{N}$ such that $(a,b)\mathtt{\sim} (c,d) $ if and only if a+d = c+b , where (a,b) and (c,d) are ordered pairs of natural numbers. The integers can be constructed as equivalence classes of these ordered pairs.

\paragraph{} Now let us define some operations on the integers using equivalent operations on the natural numbers. 
Thus
\begin{center}
        $[(a,b)] + [(c,d)] := [(a+d,c+b)]$ \\
        $[(a,b)].[(c,d)]:= [(ac+bd,ad+bc)]$\\
        $-[(a,b)]=[(b,a)]$
\end{center}

\paragraph{} Now, every such equivalence class has at least one of the type of elements (a,0) and (0,a). The classes [(a,0)] denote the Natural numbers as a subset of the integers while [(0,a)] denote the remaining integers.




\section{Mathematical induction}
\paragraph{} Let us first set up a few required tools and move on to the theorem of finite induction.

\paragraph{Well ordering Principle:} Every non-empty set S of non-negative integers contains a least element. That is,\[ \exists~ a ~ \epsilon~ S~ \textrm{such that}~~ \forall ~b~ \epsilon~ S ,  a \leq b
\] 

\begin{theorem}[\textbf{Archimedean property}]
    If a and b are any positive integers then there exists a positive integer n for which $ na ~ \ge ~ b $
\end{theorem}
\begin{proof}
    By contradiction: Let $na ~ < ~b $ for every n, for some a and b. Thus the set S defined by \\
    $ S ~ = \{b ~- na ~ | n \epsilon \mathbb{N}\}$\\
    will consist only positive integers. Therefore by the well ordering principle, it must have a least element of the form $a~-~bm$. But since $a~-~(m+1)b$  also lies in S and $(a~-~(m+1)b)~ < ~( a~-~bm)$, $ a~-~bm$ cannot be the least element. Therefore our assumption does not hold. Hence,the Archimedean property is proved true.
\end{proof}

\begin{theorem}[\textbf{Principle of finite induction}]
    Let S be the set of positive integers with the properties:\\
    (i) 1 belongs to S.\\
    (ii) For each integer k belonging to S, ~k+1~ belongs to S as well.\\
    Then, S is the set of all positive integers.
\end{theorem}
\begin{proof}
    By contradiction: Let there exist T, the set of all integers which do not belong to S. As T only has positive integers, by the well ordering principle, T must have some integer n which is the least element of T. Also as 1 belongs to S by the property (i), ~~$0 < n-1 < n ~~\textrm{as} ~ n > 1$. Therefore as n is the smallest integer not belonging to S, n-1 must belong to S. But since n-1 belongs to S, n-1~+~1~ = ~n must belong to S as well, by property (ii). Thus our assumption that T exists does not hold true. Therefore S is the set of all positive integers.
\end{proof}
\begin{remark}
    Instead of having 1 as the least element of the set S, we could have a more general theorem on the principle of induction by having some positive integer $n_0$ as the least element. 
\end{remark}




\section{Binomial theorem}

\paragraph{Binomial Coefficients:} For positive integers n and k such that $0~ \leq k ~\leq~ n ~$, $\binom{n}{k} = \frac{n!}{k!(n-k)!}$.

\paragraph{} Some identities:\\
(i) $\binom{n}{0} = \binom{n}{n} =1$\\
(ii) Pascal's rule:  $\binom{n}{k} ~ + ~ \binom{n}{k-1} ~ = ~ \binom{n+1}{k}$. This can be used to construct Pascal's triangle.

\begin{theorem}[The Binomial theorem] For $n ~\epsilon~ \mathbb{N}~ \textrm{and for a,b}~ \epsilon~ \mathbb{R}$ we have:
\\
\begin{math}
    (a+b)^n = \sum_{k=0}^{n} \binom{n}{k} a^k b^{n-k}
\end{math}
\end{theorem}
\begin{proof}
 By Induction: For n=1, the formula gives:\\
 $(a+b)^1 = \sum_{k=0}^{1} \binom{1}{k} a^k b^{1-k} = a+b$\\
 Which obviously holds.\\
 Now, assuming that the formula works for m,\\
 as $(a+b)^{m+1} = a(a+b)^m + b(a+b)^m$,\\
 Now,\\
 \begin{math}
 a(a+b)^m = \sum_{k=0}^{m} \binom{m}{k} a^{k+1} b^{m-k} = a^{m+1} +\sum_{k=1}^{m} \binom{m}{k} a^{k} b^{m-k+1}\\
 b(a+b)^m = \sum_{k=0}^{m} \binom{m}{k} a^k b^{m-k+1} = b^{m+1} + \sum_{k=1}^{m} \binom{m}{k-1} a^{k} b^{m-k+1}\\
 \\
 \textrm{Thus adding the two equations and using Pascal's rule,}\\
 (a+b)^{m+1} = a^{m+1} + \sum_{k=1}^{m} ( \binom{m}{k-1} + \binom{m}{k}) a^{k} b^{m-k+1} + b^{m+1}\\
 Thus, ~ (a+b)^{m+1} =  \sum_{k=0}^{m+1} \binom{m+1}{k} a^k b^{m+1-k}\\
 \end{math}
 \end{proof}





\chapter{Divisibility theory}

\section{The Division Algorithm} 
\begin{theorem}[The Division algorithm]
    For any integers a and b, such than b $>$ 0, there exist unique integers q and r such that $ a = bq + r$ where $0 \leq r < b$.
\end{theorem}
\begin{proof}
    Let $S = \{a~-bx~| \textrm{x is an integer and }a-bx\geq 0\}$\\
    To use the well ordering principle, we must prove that S is a non-empty set, first.\\
    Since b $\geq$ 1, we have $b|a| \geq |a|$.\\
    Thus, for $x = -|a|$, we have $ a~-bx~= a~+b|a| \geq a+|a| \geq 0$.\\
    Thus, S is non-empty. Now using the well-ordering principle on S, we must have some least element r in S, such that :\\
    a - b q = r; where r $\geq$ 0.\\
    If r $\geq$ b, a -(q+1)b = r-b $\geq$ 0 and a -(q+1)b which is less than a -qb would belong to S, violating the well ordering principle. Thus, $r<b$.\\
    Uniqueness of q and r:\\
    Let $a = bq +r = bq' + r'$\\
    Thus, subtracting the two representations of a, we get:\\
    $(r - r')= b(q - q')$\\
    But, since $0 \leq r <b$ and $0\leq r' <b$,\\
    we have $-b < -r' \leq 0$\\
    Therefore, $ -b < r - r' < b$, i.e. $|r-r'|<b$\\
    Plugging this in the previous equation, \\
    $b~|q-q'|<~b$, so $|q-q'| <1$\\
    But since $ |q-q'| \geq$ 0, we have $|q-q'|=0$. Thus $ q=q'$ and $ r=r'$. 
\end{proof}


\section{Greatest Common Divisor}
\begin{definition}[Divisibility]
    An integer a is said to be divisible by an integer b is there exists an integer c such that a = bc.\\
    This is denoted by $b|a$. If b does not divide a, $b\nmid a$.
\end{definition}

\paragraph{Consequences of Definition 2.2.1 :}
Following are some easily provable consequences of the definition of divisibility:
\begin{enumerate}
    \item $a|0~,~1|a~,~ a|a.$
    \item $a|1 \implies ~a = \pm 1$
    \item If $a|b$ and $c|d$ then $ab|cd$.
    \item If $a|b$ and $b|c$ then $a|c$.
    \item $a|b$ and $b|a ~ \iff a = \pm b$.
    \item If $a|b$ and $b\neq 0$ then $|a| \leq |b|$.
    \item If $a|b$ and $a|c$ then $a|(bx~ + ~cy)~ \forall$ integers x,y. 
\end{enumerate}

\begin{definition}[Greatest common divisor]
    Let a and b be two integers, not both of them zero. The greatest common divisor of a and b, denoted by gcd(a,b) is an integer d which satisfies the following properties:
    \begin{enumerate}
        \item $d|a$ and $d|b$.
        \item For each integer c satisfying $c|a$ and $c|b$, $d \geq c$.
    \end{enumerate}
\end{definition}

\begin{theorem}
    For any integers a and b, not both zero, there exist integers x and y such that:\\
    $gcd(a,b) ~=~ ax ~+~ by$
\end{theorem}
\begin{proof}
    Let $S ~=~ \{ au ~+~ bv | au ~+~ bv>0; u,v \textrm{ integers} \}$
    As S isn't empty, we can apply the well ordering principle to get some integer d which is the least element of S.\\
    Now by the division algorithm, for some q and r, \\
    $ a = qd ~+~ r$\\
    $ r = a ~-~ qd ~=~ a ~-~ (ax+by)q ~=~ a(1-qx) ~+~ b(-qy) $\\
    Thus, r would be a part of S too, if $r>0$. Thus r=0.
    So, $d|a$. By a similar argument, $d|b$. But as $d ~=~ ax ~+~ by$, by the 7th consequence, any other divisor c would divide d. So as $c|d$, $ c<d$. Thus d is the greatest common divisor of a and b.
\end{proof}
\begin{corollary}
    If a,b are two integers not both zero, then the set \\
    $T ~=~ \{ ax ~+~ by | x,y \textrm{integers} \}$
    is precisely the set of all multiples of gcd(a,b) =d.
\end{corollary}

\begin{definition}[Relatively prime numbers]
    Two integers a and b are said to be relatively prime in=f gcd(a,b) = 1.
\end{definition}
\begin{theorem}
    Two numbers a and b are relatively prime if and only if there exist integers x and y such that ax + by = 1.
\end{theorem}
\begin{proof}
    Direct consequence of Theorem 2.2.1 and as Corollary 2.2.1.1
\end{proof}

\begin{corollary}
     If gcd(a,b)=d then gcd(a/d,b/d)=1.
\end{corollary}
\begin{corollary}
    If $a|c$ and $b|c$ and gcd(a,b)=1 then $ab|c$
\end{corollary}

\begin{theorem}[Euclid's Lemma]
    If $a|bc$ and gcd(a,b) =1 then $a|c$
\end{theorem}
\begin{proof}
    Since gcd(a,b)=1, for some integers x and y, we have:\\
    1 = ax + by\\
    Thus, c = ca x + cb y\\
    Since $a|bc$, $a|(cax + bcy)$, and thus $a|c$
\end{proof}

\begin{lemma}
    If a = bq + r then gcd(a,b) = gcd(b,r)
\end{lemma}
\begin{proof}
    If d=gcd(a,b) then $d|a$ and $d|b$. Thus, $d|(a-qb)$ or $d|r$. In other words, d is a common divisor of b and r.If c is another common divisor of b and r, $c|(bq + r)$ or $c|a$. Therefore by definition, $c \leq d$. Thus gcd(b,r) =d.
\end{proof}

\paragraph{Euclid's algorithm} As a consequence of Lemma 2.2.4, we have:\\
$gcd(a,b) = gcd(b,r_1) = \dots = gcd(r_{n-1},r_n)=gcd(r_n,0) = r_n$\\
Thus, the gcd of a and b can be found in finite steps.
\begin{remark}
    The number of steps required to find the gcd by Euclid's algorithm is at most 5 times the number of digits of the smaller number.
\end{remark}

\begin{theorem}
    If $k>0$ then gcd(ka,kb)=k*gcd(a,b)
\end{theorem}
\begin{proof}
    Multiply each of the equations in Euclid's algorithm by k.
\end{proof}

\begin{definition}[Least common multiple]
    The least common multiple of two non-zero integers a and b is an integer c which satisfies the following properties:\\
    \begin{enumerate}
        \item $a|c$ and $b|c$
        \item If $a|p$ and $b|p$ then $c \leq p$
    \end{enumerate}
\end{definition}

\begin{theorem}
    For two integers a, b we have gcd(a,b) lcm(a,b) =ab
\end{theorem}
\begin{proof}
    Let d = gcd(a,b). So, a =dp and b=dq for some p and q. \\ Let m = ab/d = aq = bp. \\
    As we know, d = ax + by.\\
    Let c be a common multiple of a and b. So, c=au = bv.\\
    Now, $\frac{c}{m}$ = $\frac{cd}{ab}$ = $\frac{c(ax + by)}{ab}$ = $\frac{c}{b}$x + $\frac{c}{a}$y .\\
    Thus, m|c, and by definition m is lcm(a,b).
\end{proof}

\section{The Diophantine equation}
\paragraph{} Any equation with one or more variables to be solved in integers is called a Diophantine equation. A linear Diophantine equation in two variables x,y is :\\
ax + by = c.

\begin{theorem}
    The linear Diophantine equation ax + by =c has a solution if and only if $d|c$ where d= gcd(a,b) and if $x_0$,$y_0$ is a particular solution then the general solution to the equation is given by :\\
    x = $x_0$ +$\frac{bt}{d}$ and y = $y_0$ - $\frac{at}{d}$ for all integers t.
\end{theorem}
\begin{proof}
    Since d = gcd(a,b), a = dr and b = ds for some r,s.
    Thus, if c = ax + by = d( sx + ry)\\
    Therefore, $d|c$ for a solution to exist.\\
    Now, let $x_0$,$y_0$ be a solution to the equation.\\
    Thus, ax + by = c = a$x_0$ + b$y_0$\\
    Subtracting, a($x_0$ -x)=b(y-$y_0$)\\
    Thus as a = dr and b = ds, r($x_0$ -x)=s(y-$y_0$) where gcd(r,s)=1\\
    Now, since $r|s(y-y_0)$ and gcd(r,s)=1, $r|y-y_0$\\
    Therefore $y-y_0 ~=~ rt$ for some t. So, $x_0 -x ~=~ rt$
    Hence, proved.
\end{proof}
\begin{corollary}
    If gcd(a,b) =1 and $x_0,y_0$ is a particular solution of the Diophantine equation ax + by = c, then all solutions are given by x = $x_0$ +bt, y = $y_0$ - at where t is any integer.
\end{corollary}



\chapter{Primes}
\section{The fundamental theorem of Arithmetic}
\begin{definition}[Prime numbers]
    An integer p ($>1$) is called a prime number if its only positive divisors are 1 and p itself. A number greater than 1 which is not prime is termed composite. 
\end{definition}
\begin{theorem}
    If p is a prime number and $p|ab$ then either $p|a$ or $p|b$.
\end{theorem}
\begin{proof}
    If $p|a$, the statement is satisfied. So when $p \nmid a$, gcd(p,a)=1. Thus by Euclid's Lemma, as $p|ab$ and gcd(a,p)=1, $p|b$.
\end{proof}
\begin{corollary}
    If p is a prime and $p|a_0a_1a_2 \dots a_n$, then $p|a_k$ for some $1 \leq k \leq n$
\end{corollary}
\begin{corollary}
    If $p,q_0,q_1,q_2\dots ,q_n$ and $p|q_0q_1q_2\dots q_n$ then $p=q_k$ for some k such that $1\leq k \leq n$.
\end{corollary}

\begin{theorem}[The fundamental theorem of Arithmetic]
    Every positive integer n $>$ 1 can be uniquely represented in an unordered product of primes.
\end{theorem}
\begin{proof}
    If n is prime the statement holds true. If n is composite, it will have an integer d such that $d|n$ for some d, by definition. All such integers are positive and the set of these integers is non-empty. Thus by the well ordering principle, there exists a minimum divisor of n, $p_1$. As this doesn't have any further divisor and is minimum, it must be prime. We now use the same argument on $n_1 = n/p_1$, $n_2=n_1/p_2$ and so on. Now since $n>n_1>n_2> \dots >1$, we must have finite steps and the expression of $n=p_1p_2\dots p_k$ would hold. Uniqueness is a direct consequence of corollary 3.1.1.2.
\end{proof}
\begin{corollary}
    Any positive integer $n>1$ can be expressed in the form \[
    n=p_1^k_1 p_2^k_2 p_3^k_3 \dots p_m ^k_m \]
    Where for each i such that $1 \leq i \leq m$, $k_i$ is a positive integer and $p_i$ is a prime.
\end{corollary}

\section{The sieve of Eratosthenes}
     \paragraph{} For any composite number a $>$1 we have a=bc for some integers b and c by definition. Assuming $b\leq c$ WLOG, we have $b^2 \leq bc =a$. Thus $b \leq \sqrt{a}$. As b has at least one prime factor, (let p), $p \leq \sqrt{a} $. Eratosthenes devised an algorithm to find all primes from 1 to any number N. Arrange the numbers in increasing order, and start moving from 1 to N. Now whenever a prime, say p, is encountered, cross out all multiples of p till N. When you reach $x \geq \sqrt{N}$, you will have crossed out all composite numbers till N.
    \begin{theorem}
        \textbf{There are an infinite number of primes}
    \end{theorem}
        \begin{proof}[Euler's proof]
            Let $p_1,p_2,p_3 \dots p_n$ be all the primes, in ascending order. Now consider : \[ P = p_1 p_2 p_3 \dots p_n +1 \]
            Since $P>1$, we must have some prime q which divides P. But q must be one of $p_1, p_2, p_3 \dots ,p_n$. Thus, $q|P$ and $q|p_1p_2p_3\dots p_n$. Combining these two, we have $q|(P-p_1p_2p_3\dots p_n)$, and thus $q|1$ which cannot be possible.
        \end{proof}
            \begin{remark}
                $p_{n+1} \leq p_1 p_2 p_3 \dots p_n +1 \leq p_n^n$
            \end{remark}
        \begin{proof}[Alternate proof]
            Let $n_1 = 2$ and $n_i = n_1n_2n_3 \dots n_{i-1} +1$ for all positive integers i. Since each $n_i > 0$, each of them is divisible by atleast one prime. Let d = gcd($n_i,n_j$) where $i<j$. Thus, $d|n_i$ and $d|n_k$ or $d|(n_1n_2\dots n_i \dots n_k +1)$. Thus $d|1$. Therefore d = 1. Thus all $n_i$ are pairwise relatively prime and there exist as many primes ass $n_i$, that is, infinite.
        \end{proof}
    \begin{theorem}
        If $p_n$ is the $n^{th}$ prime number, then $p_n \leq 2^{2^{n-1}}$
    \end{theorem}    
        \begin{proof}
            As the given assertion is true when n=1, let it be true for all integers upto n. Now, 
            \[ p_{n+1} \leq  p_1 p_2 p_3 \dots p_n +1 \leq 2.2^2.2^3 \dots 2^n +1\]
            Now since $1 \leq 2^{2^n -1}$ we have 
            \[ p_{n+1} \leq  2^{2^n-1} + 2^{2^n-1} = 2^{2^n}\]
        \end{proof}
        \begin{corollary}
            There are at least n+1 prime less than $2^{2^n} $ for $n\geq 1$.\\
        \end{corollary}
\newpage

\section{The Goldbach Conjecture}
    \paragraph{The Goldbach conjecture:} Every even integer is the sum of two numbers that are either 1 or primes. In other words, every odd number n  greater than 7 can be expressed as the sum of 3 primes, as n -3 is even.  \begin{theorem}
        There is an infinite number of primes of the form 4n +3.
    \end{theorem}
        \begin{proof}
            Let there be only finite primes of the form 4n +3, given by $q_1,q_2,\dots q_n$. Let 
            \[ N = 4q_1q_2q_3\dots q_n -1 = 4(q_1q_2q_3\dots q_n -1) +3 \]
            Let $r_1,r_2,r_3, \dots r_k$ be the prime factors of N. Since N is of the form 4n +3 too, 2 is not a factor of N. Thus each $r_i$ is either of the form 4n +1 or 4n +3. But product of two numbers of the form 4n +1 is also of the form 4n +1. Therefore each prime  factor of N is of the form 4n +3, and must belong to $q_1,q_2,\dots q_n$. But that would mean $r_i|1$. Thus the assumption does not hold and there are infinite primes of the form 4n +3.
        \end{proof}
    \begin{theorem}
        If a and b are relatively prime positive integers, then the arithmetic progression \[a, a + b, a + 2b, a + 3b \dots \] has infinitely many primes.\\
        But, there exists no infinite arithmetic progression consisting solely of primes.
    \end{theorem}    





\chapter{Congruences}
    \section{Basic properties of congruence}
        \begin{definition}
            Let n be a positive integer. Two integers a and b are said to be congruent modulo n if n divides the difference a - b, that is, a - b = kn for some integer k. This is denoted by: \[ a \equiv b(mod n)\]
            The set of integers 0,1,2,$\dots$,n-1 is called the set of least positive residues modulo n.
        \end{definition}
        \begin{theorem}
            For arbitrary integers a and b, $a \equiv b( mod ~n)$ if anf only is a and b leaave the same remainder on dividing by n.
        \end{theorem}
            \begin{proof}
                When $a \equiv b(mod~ n)$ we have a = b + kn. Now if b = qn +r for some q and r such that $0 \leq r < n$, we have a = qn +r + kn = (q+k)n +r. Thus a leaves the same remainder as b on dividing by n.
                Now, if $a=q_1n ~+~ r$ and $b=q_2n ~+~ r$, we have a-b = ($q_1 - q_2$)n, thus $ a\equiv b$.
            \end{proof}
            \newpage
        \paragraph{Some properties} Let a,b,c,d be arbitrary integers and n be an integer. Then:
            \begin{enumerate}
                \item $a\equiv a(mod ~n)$
                \item $a\equiv b(mod ~n) \iff b\equiv a(mod ~n)$
                \item $a\equiv b(mod ~n) \textrm{ and } b\equiv c(mod ~n) \implies a\equiv c(mod ~n) $
                \item  If $a\equiv b(mod ~n) \textrm{ and } c\equiv d(mod ~n) \textrm{ then } a+c\equiv b+d(mod ~n) \textrm{ and } ac\equiv bd(mod ~n)$
                \item If $a\equiv b(mod ~n) \textrm{ then } a+c\equiv b+c(mod ~n) \textrm{ and } ac\equiv bc(mod ~n)$ 
                \item If $a\equiv b(mod ~n)$ then $a^k\equiv b^k(mod ~n)$ for any positive integer k.
            \end{enumerate}
        \begin{theorem}
            If $ca\equiv cb(mod ~n)$ then $a\equiv b(mod ~n/d)$ where d = gcd(n,c).
        \end{theorem}  
            \begin{proof}
                As ca - cb = c(a - b) = kn for some integer k, and d = gcd(c,n), we have c = dr and n = ds for some integers r and s. Thus, r(a - b) = s. As gcd(r,s) =1, we thus have that $s|(a-b)$ or in other words, $a\equiv b(mod ~s)$.
            \end{proof}
            \begin{corollary}
                If $ca\equiv cb(mod ~n) $ and gcd(c,n) =1, we have $a\equiv b(mod ~n)$.
            \end{corollary}
    \section{Special Divisibility tests}
        \paragraph{Place value notation of numbers} 
            Given an integer b $>1$, any integer N can be written as a unique combinations of powers of b as \[ N = a_m b^m + a_{m-1} b^{m-1} \dots a_1 b + a_0\]
            Or in other form, \[ N = (a_m a_{m-1} \dots a_1 a_0)_b \]
            \begin{proof}
                The division algorithm yields $ N ~=~ q_0 b ~+~ r$ where $0\leq r <b$. Now similarly it will yield $q_0 ~=~ q_1 b ~+~ r$ and so on. As $N ~>~ q_0 ~>~ q_1 \dots  >0$, it is a decreasing sequence of integers and must terminate. Thus, let the last of $q_i$ be $q_m$. Now substituting all of $q_i$'s in\\ N = $q_0$b + r, we get the required form. 
                For uniqueness, let $a_i$ and $c_i$ be the two distinct sets of coefficients of $b^i$ for all i from 0 to m, then subtracting the representations of N, we get \[ 0 = (a_m - c_m)b^m + \dots + (a_0 - c_0) \]
                Let $d_i = a_i - c_i$ and $d_k $ be the non-zero $d_i$ for the smallest subscript k. Now, \[ 0 = d_m b^m + \dots d_k b^k\]
                \[ d_k = -b(d_m b^{m-k-1} + \dots + d_{k+1} )\]
                Thus $b|d_k$. But, $|d_k| <b$ as $|a_k - c_k| < b$. Thus, we have reached contradiction.
            \end{proof}
        \begin{theorem}
            Let P(x) = $\sum_{m=0}^{k} c_l x^k$ be a polynomial in x, and if $a\equiv b(mod ~n)$ then $P(a)\equiv P(b)(mod ~n)$
        \end{theorem}    
            \begin{proof}
                Direct consequence of properties discussed earlier.
            \end{proof}
            \begin{corollary}
                If a is a solution of $P(x) \equiv 0 (mod ~n)$ and $a\equiv b(mod ~n)$ then b is also a solution.
            \end{corollary}
        \begin{theorem}
            Let N = $\sum_{k=0}^{m} a_k 10^k$ and S = $\sum_{k=0}^{m} a_k$ then we have $9|N$ only if and only if $9|S$.
        \end{theorem}    
            \begin{proof}
                We can prove this by simply using the previous theorem and the fact that $10\equiv 1(mod ~9)$
            \end{proof}
        \begin{theorem}
            Let N = $\sum_{k=0}^{m} a_k 10^k$ and T = $\sum_{k=0}^{m} a_k(-1)^k$ then we have $11|N$ only if and only if $11|T$.
        \end{theorem}    
            \begin{proof}
                We can similarly prove this theorem by using a similar argument as the previous, and the fact that \\ $ 10 \equiv (-1) (mod ~11) $ .
            \end{proof}
    \section{Linear Congruences}
        \paragraph{} An equation of the form ax $\equiv$ b (mod n) is called a linear congruence. A solution of this equation is an integer $x_0$ for which a$x_0 \equiv$ b (mod n). This, by definition, means $a x_0 - b = n y _0$ for some $y_0$. This reduces to the diophantine equation $ a x_0~ -~ n y_0 ~=~ b$.
        \begin{theorem}
            The linear congruence x $\equiv$ b (mod n) has a solution if and only if $d|b$ where d = gcd(a,n). In this case, it has d mutually congruent sets of solutions.
        \end{theorem}
            \begin{proof}
                Recall results of Diophantine equation's solution analysis from chapter 2.
            \end{proof}
            \begin{remark}
                Here the d mutually incongruent solutions are: \[ x_0, x_0 + n/d , x_0 + 2(n/d) \dots , x_0 + (d-1)n/d \]
            \end{remark}
            \begin{corollary}
                If gcd(a,n) =1 then the linear congruence ax $\equiv$ b (mod n) has a unique solution modulo n.
            \end{corollary}
            
        \begin{theorem}[The Chinese remainder theorem]
            Let $n_1, n_2, n_3 \dots n_r$ be positive integers such that gcd($n_i,n_j$) =1 $\forall$ $i\neq j$. Then the system of linear congruences :
            \[ x \equiv a_1 (mod n_1)\]
            \[ x \equiv a_2 (mod n_2)\]
            \[ x \equiv a_3 (mod n_3)\]
            \[.\]
            \[.\]
            \[ x \equiv a_r (mod n_r)\]
            has a simultaneous solution which is unique modulo $(n_1 n_2 n_3 \dots n_r)$.
        \end{theorem}
            \begin{proof}
                Let n = $n_1 n_2 n_3 \dots n_r$\\
                Now, for each i such that $1\leq i \leq r$, let $N_i = n/n_i$. By hypothesis, gcd($N_i,n_i$) =1. Thus a solution for the linear congruence $ N_i x \equiv 1$ (mod $n_i$) exists, let that be $x_i$.\\
                Let $\Bar{x} = a_1 N_1 x_1 + \dots + a_r N_r x_r$.\\
                As $N_i \equiv 0$(mod $n_k$) for k $neq$ i. Thus, $\Bar{x} \equiv a_k N_k x_k (mod ~n_k)$. But since $N_k x_k \equiv 1 (mod ~ n_k)$ we have \\
                $\Bar{x} \equiv a_k . 1 (mod ~n_k)$ for all k\\
                Thus a simultaneous solution exists.\\
                If there is another solution $x'$ satisfying the given system of linear congruences, then $n_k|(\Bar{x} - x')$ for each k.\\ 
                Thus $n_1 n_2 n_3 \dots n_r |(\Bar{x} - x')$.\\
                Therefore $\Bar{x} \equiv x' (mod ~n)$.
            \end{proof}
            
    


\chapter{Fermat's theorem}
    \section{Fermat's factorization method}
        \paragraph{} For any odd integer n, finding factors n = ab is equivalent to solving for x and y in n = $x^2 - y^2$, where $x = (a + b)/2$ and $y=(a-b)/2$. For even integers, powers of 2 can be separated into factors and thus odd integers are our concern.\\
        Now for $x^2 -n = y^2$, we need to determine the smallest integer k such that $k^2 \geq n$. Now the numbers \[ k^2 -n , (k+1)^2 -n , \dots \] are to be searched for perfect squares, until the trivial solution of n=n.1, or \[ ((n+1)/2)^2 - n = ((n-1)/2)^2 \] is reached.
    
    \section{Fermat's Little Theorem}
        \begin{theorem}
            If p is a prime and $ p \nmid a $ then $a^{p-1} \equiv 1 $(mod p).
        \end{theorem}
            \begin{proof}
                Consider the set of integers a, 2a, 3a, $\dots$ (p-1)a. None of these is congruent modulo p to any other, or 0. Also, this set of integers is congruent modulo p to 1, 2, 3, $\dots$ ,p-1 in some order. Multiplying these congruences, we get: 
                \[ a^{p-1} (p-1)! \equiv (p-1)! (mod p)\]
                But as gcd(p,(p-1)!) =1, we can cancel (p-1)! on both sides and reach our result.
            \end{proof}
            \begin{corollary}
                If p is prime, $a^p \equiv a$(mod p) for any integer a. 
            \end{corollary}
            \begin{lemma}
                If p and q are distinct primes, $a^{pq} \equiv a $(mod pq).
            \end{lemma}
                \begin{proof}
                    Use $a^q$ in the above corollary to obtain\\ $a^{pq} \equiv a$(mod p). Similarly, $a^{pq} \equiv a$(mod q). Therefore $a^{pq} \equiv a$(mod pq).
                \end{proof}
    \section{Wilson's theorem}
        \begin{theorem}
            If p is prime, then (p-1)! $\equiv$ (-1) (mod p).
        \end{theorem}
           \begin{proof}
               As p=2 and p=3 are obvious cases, let us take $p>3$. Now consider the integers 1,2,3,4 $\dots$,p-1.
               Let us consider the linear congruence ax $\equiv$ 1(mod p). But gcd(a,p) =1. Thus a solution $a'$ exists such that\\ $1\leq a' \leq p-1$.
               Since p is prime, a = $a'$ only is a =1 or a = p -1, since the congruence $a^2 \equiv 1(mod ~p)$ is nothing but $(a^2 -1) \equiv 0(mod ~p)$ or either $a - 1 \equiv 0(mod ~p)$ where a =1 or $a +1 \equiv 0(mod ~p)$ where a =p -1.
               Thus grouping the remaining numbers from the earlier list into (p-3)/2 pairs, and multiplying the congruences, we get: \[ 2.3.4\dots (p-2) \equiv 1(mod p) \]
               Thus multiplying by p -1. \[ (p-1)! \equiv (p-1) (mod ~p) \equiv (-1) (mod ~p) \]
            \end{proof}
        \begin{theorem}
            The quadratic congruence $x^2 +1 \equiv 0(mod ~p)$ where p is an odd prime, has a solution only if $p \equiv 1(mod ~4)$ 
        \end{theorem}   
            \begin{proof}
                Let a be a solution of the quadratic congruence, so that $a^2 +1  \equiv 0(mod ~p)$ \\
                Therefore, $a^2 \equiv (-1) (mod ~p)$\\
                But since $p \nmid a$, by Fermat's theorem we have
                \[ 1 \equiv a^{p-1} \equiv (a^2)^{\frac{p-1}{2}} \equiv (-1)^{\frac{p-1}{2}} (mod ~p) \]
                As $1 \equiv -1 (mod ~p)$ is false, p cannot be of the form\\ 4n + 3. Thus, p is of the form 4n + 1.
            \end{proof}


\chapter{Number theoretic functions}
    \section{The functions $\tau $  and $\sigma$}
        \paragraph{} Any function whose domain of definition is the set of positive integers is called as a number theoretic function. 
        \begin{definition}
            For any positive integer n, $\tau$(n) denotes the number of divisors of n and $\sigma$(n) denotes the su of all divisors of n.\\
            Here,
            \[ \tau(n) = \sum_{d|n} 1 \]
            \[ \sigma(n) = \sum_{d|n} d \]
        \end{definition}
        \begin{theorem}
            If n = $p_1^{k_1} p_2^{k_2} p_3^{k_3} \dots p_r^{k_r}$ is the prime factorization of an integer n $>$ 1, then the positive divisors of n are precisely those integers d of the form \[ d = p_1^{a_1} p_2^{a_2} p_3^{a_3} \dots p_r^{a_r} \] 
            where $0 \leq a_i \leq k_i$ for all $1 \leq i \leq r$.
        \end{theorem}
            \begin{proof}
                Let d = $q_1 q_2 \dots q_s$ and $d' = t_1 t_2 \dots t_x$ be the divisors of n, where $q_i$ and $t_i $ are prime, such that n = $dd'$. Thus, \[ n = p_1^{k_1} p_2^{k_2} p_3^{k_3} \dots p_r^{k_r} = q_1 q_2 \dots q_s t_1 t_2 \dots t_x \]
                Now by uniqueness of the prime factor representation,  Each $q_i$ must be one of $p_k$. Thus, d = $p_1^{a_1} p_2^{a_2} p_3^{a_3} \dots p_r^{a_r}$ . Now, $d' = n/d$. The exponents of primes being positive in $d'$ gives the condition that $a_i < k_i$ .
            \end{proof}
        \begin{theorem}
            If n = $p_1^{k_1} p_2^{k_2} p_3^{k_3} \dots p_r^{k_r}$ is the prime facrorization of an integer $n>1$ then:
            \[ 1. \tau(n) = (k_1 +1)(k_2 +1)\dots(k_r +1)\]
            \[ 2. \sigma(n) = \frac{p_1^{k_1 +1}-1}{p_1 -1}  \frac{p_2^{k_2 +1}-1}{p_2 -1} \dots  \frac{p_r^{k_r +1}-1}{p_r -1} \]
        \end{theorem}    
        \begin{remark}
             The product of all divisors of n is $n^{\frac{\tau(n)}{2}}$.
        \end{remark}
        \begin{definition}
            A number theoretic function f is said to be multiplicative if \[f(mn) = f(m)f(n)\]
            where gcd(m,n) = 1.
        \end{definition}
        \begin{remark}
             The funtions $\tau$ and $\sigma$ are multiplicative.
        \end{remark}
        \begin{lemma}
            If gcd(m,n) =1, then the set of positive divisors of mn consist of all the numbers $d_1 d_2$ such that $d_1|m$ and $d_2|n$ and gcd($d_1 , d_2$ ) =1. Also, all these products are distinct. 
        \end{lemma}
            \begin{proof}
                Trivial using prime factorization and theorem 6.1.1.
            \end{proof}
        \begin{theorem}
            If f is a multiplicative function and F is a funtion defined by \[ F = \sum_{d|n} f(d) \] then F is also multiplicative.
        \end{theorem}    
            \begin{proof}
                Let m, n be two positive integers. Therefore, \[ F(mn) = \sum_{d|mn} f(d) = \sum_{d_1|m , d_2|n} f(d_1 d_2) \]
                By the previous Lemma and multiplicativity of f, we have:
                \[ F(mn) = \sum_{d_1|m , d_2|n } f(d_1)f(d_2) = (\sum_{d_1|m} f(d_1))(\sum_{d_2|n} f(d_2)) = F(m)F(n) \]
            \end{proof}
    \section{The Mobius inversion formula}
        \begin{definition}[The Mobius function $\mu$]
            For any integer n, define \\ $ \mu(n) = 1 \textrm{ if } n=1 ;~ \\ \mu(n) = 0 \textrm{ if } p^2|n\textrm{ for some prime p ; }\\ \mu(n) = (-1)^r \textrm{ if } n= p_1 p_2 p_3 \dots p_r\textrm{ where }p_i\textrm{ are primes.}$
        \end{definition}
        \begin{remark}
             $\mu$ is a multiplicative function.
        \end{remark}
        \begin{theorem}
            For any integer n $\geq 1$, we have :\\
            $\sum_{d|n} \mu(d)$ = 1 if n =1\\
            $\sum_{d|n} \mu(d)$ = 0 is $n >1$\\
        \end{theorem}
        \begin{theorem}[Mobius inversion formula]
            Let f and F be two number theoretic functions related by the formula :\[ F(n) = \sum_{d|n} f(d) \]
            Then, \[f(n) = \sum_{d|n} \mu(d) F(n/d) \]
        \end{theorem}
            \begin{proof}
                \[ \sum_{d|n} \mu(d) F(n/d) = \sum_{d|n} \mu(d) \sum_{c|(n/d)} f(c) = \sum_{d|n}  \sum_{c|(n/d)} \mu(d)f(c)\]
                As $d|n$ and $c|(n/d)$, we have $c|n$ and $d|(n/c)$. Thus, 
                \[  \sum_{c|(n/d)} \mu(d)f(c) = \sum_{c|n} (\sum_{d|(n/c)} \mu(d)f(c))  =  \sum_{c|n} f(c) (\sum_{d|(n/c)} \mu(d)) \]
                But, by theorem 6.2.1, $\sum_{d|(n/c)} \mu(d)$ is non-zero only when n/c =1, i.e. n=c. Thus, \[\sum_{c|n} f(c) (\sum_{d|(n/c)} \mu(d)) = f(n) \]
            \end{proof}
            \begin{remark}
                 \[ 1 = \sum_{d|n} \mu(n/d) \tau(d) \]
                 \[ n = \sum_{d|n} \mu(n/d) \sigma(d) \]
            \end{remark}
    \section{The greatest integer function}
        \begin{definition}
            For any real number x, define [x] to be the largest integer less than or equal to x. In other words, [x] is the unique integer satisfying $x-1<[x]<x$.
        \end{definition}
        \begin{theorem}
            If n is a positive integer and p is a prime, then the exponent of p in the prime factorization of n! is \[ \sum_{k=1}^{\infty} [n/p^k]\]
        \end{theorem}
        \begin{theorem}
            Let f and F be number theoretic functions such that
            \[ F(n) = \sum_{d|n} f(d) \]
            Then, for any integer N,
            \[ \sum_{n=1}^{N} F(n) = \sum_{k=1}^{N} f(k)[N/k] \]
        \end{theorem}
            \begin{proof}
                \[\sum_{n=1}^{N} F(n) = \sum_{n=1}^{N} \sum_{d|n} f(d) \]
                For a fixed integer k $\leq$ N, we have $[N/k]$ multiples of k occuring from 1 to N, and thus the function f will be evaluated  $[N/k]$ times. Summing over, we get the desired result.
            \end{proof}
            \begin{corollary}
                If N is a positive integer,
                \[ \sum_{n=1}^{N} \tau(n) = \sum_{n=1}^{N}[N/n] \]
                \[ \sum_{n=1}^{N} \sigma(n) = \sum_{n=1}^{N} n[N/n] \]
            \end{corollary}


\chapter*{Revised plan of action}
  \paragraph{} Following is my revised Plan of action:
  \begin{enumerate}
        \item Week 5- Chapter 7: Euler’s generalization of Fermat’s theorem.
        \item Week 6- Chapter 8: Primitive roots and indices.
        \item Week 7- Chapter 9: The quadratic reciprocity law ,Chapter 10: Perfect numbers.
        \item Week 8- Chapter 11: The Fermat conjecture ,Chapter 12: Representation of integers as sum of squares.
        \item Extra- Chapter 13: Fibonacci numbers and continued fractions.
  \end{enumerate}
\end{document}
